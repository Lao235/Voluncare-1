
\documentclass[12pt]{amsart}
\usepackage{geometry} % see geometry.pdf on how to lay out the page. There's lots.
\usepackage{graphicx} 
\geometry{a4paper} % or letter or a5paper or ... etc
% \geometry{landscape} % rotated page geometry





% See the ``Article customise'' template for come common customisations


\title{}
\author{}
\date{} % delete this line to display the current date


\makeatletter %?\section???????
\renewcommand{\section}{\@startsection{section}{1}{0mm}
  {-\baselineskip}{0.5\baselineskip}{\bf\leftline}}
\makeatother



%%% BEGIN DOCUMENT
\begin{document}
\section{Technical Problems}


\subsection{\textbf{Text editor technology}}
\subsubsection{\textbf{LaTex}}
\paragraph{LaTax is a better text editor comparing to Word in academic writing. We can only concentrate on content with LaTex, because it provides structures with command such as section and the format of the text it produced is much more beautiful. In addition, it is convenient in producing catalog and reference. Besides, our supervisor prefers the documentation in a format produced by LaTex. But we haven't used it before, to solve this, we organized a workshop to study it.}



\subsection{\textbf{Peer coding technology}}
\subsubsection{\textbf{Github}}
\paragraph{In order to maintain the consistence of our group project code, it is important for every one to understand code written by others, this making us choose Github for peer coding.
However, this is the first experience of peer coding for us. To solve this, we learn how to use Github online ourselves.}



\subsection{\textbf{coding technology}}
\subsubsection{\textbf{MEAN+Ionic}}
\paragraph{MEAN is the abbreviation of MongoDB, Express, AngularJS and Node.js. AngularJS and Ionic are front-end technology, considered as one of the best frameworks about mobile web app and fitting our project requirement perfectly. MongoDB, NodeJS and Express are back-end technology. MongoDB is a kind of database provides is fast in query access by using query language of MongoDB, whose data model is flexible, evolving with the change of requirements from our stakeholder. NodeJS and Express are used to connect with MongoDB and front-end. The problem is that we have't used this before and the concept of it is quit different, and our stakeholder gave us a little time to complete first version of the product. As a result, we mis-evaluates time-consuming of each task in our first sprint of Scrum. We realized this at the middle of our first sprint, and then, the main task in the rest of the first sprint is changed to study the new technology.}







\subsubsection{\textbf{Reference}}
\paragraph{http://moduscreate.com/5-best-mobile-web-app-frameworks-ionic-angularjs/}
\paragraph{https://www.sitepoint.com/10-reasons-use-angularjs/}
\paragraph{https://www.mongodb.com/compare/mongodb-mysql}

\end{document}